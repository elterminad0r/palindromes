\documentclass{article}
\title{Palindromes assignment}
\author{Izaak van Dongen}

\usepackage{savetrees}

\usepackage{amsmath}
\usepackage{amsfonts}
\usepackage{commath}

\usepackage[parfill]{parskip}

\usepackage[utf8]{inputenc}

\usepackage{graphicx}

\graphicspath{ {images/} }

\usepackage{listings}
\usepackage{color}

\definecolor{codegreen}{rgb}{0,0.6,0}
\definecolor{codegray}{rgb}{0.5,0.5,0.5}
\definecolor{codepurple}{rgb}{0.58,0,0.82}
\definecolor{backcolour}{rgb}{0.95,0.95,0.92}

\lstdefinestyle{mystyle}{
    backgroundcolor=\color{backcolour},   
    commentstyle=\color{codegreen},
    keywordstyle=\color{magenta},
    numberstyle=\tiny\color{codegray},
    stringstyle=\color{codepurple},
    basicstyle=\footnotesize,
    breaklines=true,                 
    captionpos=b,                    
    keepspaces=true,                 
    numbers=left,                    
    numbersep=5pt,                  
    showspaces=false,                
    showstringspaces=false,
    showtabs=false,                  
    tabsize=2
}
 
\lstset{style=mystyle}

\begin{document}
    \maketitle
    \tableofcontents
    \lstlistoflistings

    \section{Palindrome checker functions}
    Firstly, I will write a number of pure functions assuming that their input
    has been 'cleaned'. Formally, they expect a string of lowercased
    characters.
    \subsection{Checking from either }
    The first approach to this is to simply check if each character 'lines up'
    with its complementary character at the other end of the string.

\begin{lstlisting}[language=Pascal, caption=Nonrecursive palindrome function]
function nr_ispal(s: string): boolean;
var
    lower, upper: integer;
begin
    lower := 1;
    upper := length(s);
    while upper > lower do
        if s[lower] <> s[upper] then
            exit(False)
        else begin
            lower := lower + 1;
            upper := upper - 1;
        end;
    nr_ispal := True
end;
\end{lstlisting}

    This uses the more C-like idiom of breaking from a loop (an equivalent C
    function might look like this):

\begin{lstlisting}[language=C, caption=palindrome checking in C]


    \begin{lstlisting}[language=Pascal, caption=Full source]
program Palindromes;

function _is_palindrome(s: string; lower, upper: integer): boolean;
begin
    if lower >= upper then
        _is_palindrome := True
    else if s[lower] = s[upper] then
        _is_palindrome := _is_palindrome(s, lower + 1, upper - 1)
    else
        _is_palindrome := False;
end;

function is_palindrome(s: string): boolean;
begin
    is_palindrome := _is_palindrome(s, 1, length(s));
end;

function _clean(s: string; i: integer): string;
begin
    if i > length(s) then
        _clean := ''
    else if s[i] in ['A'..'Z', 'a'..'z'] then
        _clean := s[i] + _clean(s, i + 1)
    else
        _clean := _clean(s, i + 1);
end;

function clean(s: string): string;
begin
    clean := _clean(s, 1);
end;

function nrclean(s: string): string;
var
    i: char;
begin
    nrclean := '';
    for i in s do
        if i in ['A'..'Z','a'..'z'] then
            nrclean := nrclean + LowerCase(i);
end;

function nr_ispal(s: string): boolean;
var
    lower, upper: integer;
begin
    lower := 1;
    upper := length(s);
    while upper > lower do
        if s[lower] <> s[upper] then
            exit(False)
        else begin
            lower := lower + 1;
            upper := upper - 1;
        end;
    nr_ispal := True
end;

procedure menu;
var
    s: string;
begin
    write('Enter the word > '); readln(s);
    writeln('Palindrome?: ', is_palindrome(clean(s)),
                        ' ', nr_ispal(nrclean(s)));
end;

begin
    while true do menu;
end.
    \end{lstlisting}
\end{document}
